%------------------------------------------------------------------------------
% introduction.tex
%
% This  chapter describe the problem that the canditate must
% resolve in order to participate in final examination and the
% problematics that he encountered during the desing and the
% development.
%------------------------------------------------------------------------------
\chapter*{introduzione}
\addcontentsline{toc}{chapter}{Introduzione}
\label{introduzione}
Il presente documento contiene le considerazioni finali, tratte dallo studente, in merito al progetto relativo al Corso di Sistemi Concorrenti e Distribuiti seguito durante l'anno accademico 2014-2015 presso l'Università degli Studi di Padova. Il progetto ha lo scopo di far acquisire le abilità nel risolvere problematiche utilizzando le tecniche di base della \keyword{concorrenza} e della \keyword{distribuzione}

\section*{il problema affrontato}
\addcontentsline{toc}{section}{Il problema affrontato}
\label{introduzione-il-problema-affrontato}
Al termine del corso è stata richiesta, da parte del docente, la progettazione e la conseguente realizzazione di un simulatore di traffico cittadino.

Il sistema, oggetto della relazione, è composto da:

\begin{itemize}
\item{una città  configurabile prima dell'avvio del sistema;}
\item{un insieme configurabile di persone che, in simultanea, svolgono le proprie mansioni all'interno della città;}
\item{un insieme configurabile di veicoli che, in simultanea, sono in grado di trasportare le persone a specifici indirizzi all'interno della città; Il sistema deve possedere almeno i seguenti mezzi di locomozione:}
\begin{itemize}
\item{autobus;}
\item{automobili;}
\item{pedoni.}
\end{itemize}
\item{una componente di monitoraggio che consenta all'operatore di poter essere a conoscenza della stato attuale della simulazione.}
\end{itemize}

Al fine di rendere l'esperienza di utilizzo del simulatore più funzionale e gradevole, si è corredato il sistema di:

\begin{itemize}
\item{una componente grafica che permetta la visualizzazione della simulazione da più computer contemporaneamente;}
\item{una componente che sia in grado di configurare un eseguibile, in grado di lanciare l'intero sistema, secondo le esigenze configurate dal sistemista.}
\end{itemize}

\section*{struttura del documento}
\addcontentsline{toc}{section}{Struttura del documento}
\label{introduzione-struttura-del-documento}
Il seguente documento è stato cosi suddiviso in capitoli:

\begin{enumerate}
\item{\keyword{problematiche}: capitolo in cui sono illustrate le problematiche riscontrate durante la fase di progettazione del sistema;}
\item{\keyword{analisi della soluzione}: capitolo in cui viene eseguita una analisi della soluzione proposta;}
\item{\keyword{soluzione proposta}: capitolo in cui viene illustrata la soluzione;}
\item{\keyword{conclusioni}: capitolo in cui si sono tratte delle considerazioni finali.}
\end{enumerate}

\section*{convenzioni tipografiche}
\addcontentsline{toc}{section}{Convenzioni tipografiche}
\label{introduzione-convenzioni-tipografiche}
Nel seguente documento si sono rispettate le seguenti convenzioni tipografiche:

\begin{itemize}
\item{\keyword{grassetto}: per i termini rilevanti all'interno del paragrafo;}
\item{\english{corsivo}: per i termini in lingua inglese.}
\end{itemize}