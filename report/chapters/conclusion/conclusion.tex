%------------------------------------------------------------------------------
% conclusion.tex
%
% This illustrates the student conclusion. 
%------------------------------------------------------------------------------
\chapter*{conclusioni}
\addcontentsline{toc}{chapter}{Conclusioni}
\label{conclusioni}
L'opportunità di affrontare questo progetto ha fornito la possibilità di valutare molti aspetti inerenti al lavoro dello sviluppo \english{software}. Esso non ha ha riguardato solamente il singolo corso in sé ma ha consentito di riassumere conoscenze provenienti da altri corsi, seguiti anche durante lo svolgimento della Laurea Triennale.

Generalmente nello sviluppo \english{software} effettuato per altri corsi seguiti in precedenza, dato un obiettivo il fine ultimo consisteva solamente nel costruire un prototipo funzionante senza troppo preoccuparsi della correttezza delle scelta progettuali effettuate, cercando invece di adattarle man mano che il progetto evolveva. Con questo progetto ci si è imposti l'implementazione di \keyword{processi}, provenienti dal campo dell'ingegneria del \english{software}, che hanno sancito il seguente flusso di lavoro ciclico, a seguito di una fase di progettazione globale del sistema:

\begin{itemize}
\item{sviluppo della singola componente;}
\item{test di unità sulla componenti in casi comuni di utilizzo della medesima;}
\item{stesura della relativa documentazione, sia per il codice sorgente che per il codice di test.}
\end{itemize}

L'utilizzo di tale metodologia ha consentito di scoprire errori, inerenti alla progettazione, in fase d'implementazione delle singole componenti evitando cosi una grossa perdita di tempo dovuta alla gestione di molteplici errori rilevati nel momento dell'integrazione dell'intero sistema.

Infine, parlando di tecnologie, questo progetto mi ha consentito di vagliarne diverse prima di entrare nella fase dello sviluppo consentendo cosi una scelta delle stesse non basata sulla semplicità di utilizzo o sulla ``moda del momento'', ma bensì cercandone alcune che avessero in sé quelle caratteristiche che consentissero uno sviluppo agile del progetto.